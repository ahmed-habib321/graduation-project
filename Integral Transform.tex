\section{Integral Transformations}
\begin{figure*}[b]
    \begin{minipage}[h]{\textwidth}
        \begin{enrichment}{Pierre Simon Laplace}{Chars/Laplace.jpg}{2.4}{.8}{.17}
            Laplace (1749-1827) a prominent figure in mathematics and physics, made significant contributions to a wide range of fields during the 18th and 19th centuries. 
            Laplace's mark on fractional calculus was significant, though not all-encompassing.  He provided a crucial stepping stone in 1812 by defining a fractional derivative through an integral equation. This definition, along with subsequent work by mathematicians like Riemann and Liouville, paved the way for fractional calculus to become a robust field studying derivatives and integrals with "in-between" orders.
        \end{enrichment} 
    \end{minipage}
\end{figure*}
Integral transformations are mathematical operations that map functions from one space to another, 
typically defined on the real line or in higher dimensions. They represent a fundamental tool for solving problems and simplifying complex mathematical operations
we will take a look at Laplace,Fourier, Mellin transformations for fractional integral and fractional derivatives that are 
essential to solve a lot of differential equation 
\subsection{Laplace Transform}
The Laplace Transform of the function $f(t)$ defined by 
\begin{equation}
    \mathcal{L}\left[ f(t) \right] = \int_{0}^{\infty} e^{-st} f(t) dt = F(s)
\end{equation}
The function $F(s)$ is of complex variable $s$ for the existence of 
this integral $f(t)$ must be exponential of order $\alpha$ this means 
that $\exists M,T \in \mathbb{R}^+$ Such that
\[
e^{-\alpha t} f(t) \leq M \quad,\quad \forall t > T
\]
i.e $f(t)$ must not grow faster than a certain exponential function when $t \to \infty$

And to restore the original function $f(t)$ from $F(s)$ we use the inverse Laplace Transform
\[
f(t) = \mathcal{L}^{-1}\left[ F(s) \right] = \int_{\gamma-i\infty}^{\gamma+i\infty} e^{st} F(s) \hquad ds 
\]
Where the integration is done along the vertical line $ Re(s)=\gamma$ in the complex plane such that 
$\gamma$ is greater than the real part of all singularities of $F(s)$ and $F(s)$ is bounded on the line

For example if the contour path is in the region of convergence. If all singularities are in the left half-plane, or 
$F(s)$ is an entire function, then $\gamma$ can be set to zero.

Now some of the useful properties of the Laplace Transform that we will use later 
\setcounter{property}{0}
\begin{property}
    The Laplace Transform Of The Convolution 
    \[
    f(t)*g(t) = \int_{0}^{t} f(t-s)g(t) \hquad dt = \int_{0}^{t} f(t)g(t-s) \hquad dt
    \]
    \begin{equation}
        \mathcal{L}\left[ f(t)*g(t) \right] = \mathcal{L}\left[ f(t) \right]\mathcal{L}\left[ g(t) \right] = F(s)G(s)
    \end{equation}
\end{property}
\begin{property}
The Laplace Transform Of The Derivative Of An Integer Order $n$
\begin{equation}
    \mathcal{L}\left[ f^{(n)}(t) \right] = s^n F(s) - \sum_{k=0}^{n-1} s^{n-k-1} f^{(k)}(0)
\end{equation}
OR
\begin{equation}
    \mathcal{L}\left[ f^{(n)}(t) \right] = s^n F(s) - \sum_{k=0}^{n-1} s^k f^{(n-k-1)}(0)
\end{equation}
Which can be obtained from (4.1) by integrating by parts 
\end{property}
\subsubsection{Laplace Transform Of Riemann-Liouville Fractional Integral}
We will start with Laplace Transform of Riemann-Liouville fractional Integral
\[
    I^\alpha f(t) = \frac{1}{\Gamma(\alpha)}\int_{0}^{t} (t-s)^{\alpha-1}f(s) \hquad ds
\]
We know that we can rewrite it as a convolution
\[
    I^\alpha f(t) = \frac{1}{\Gamma(\alpha)} g(t)*f(t)
\]
Where $\displaystyle g(t) = t^{\alpha-1}$

Now take the Laplace Transform for both sides
\[
    \mathcal{L}\left[ I^\alpha f(t) \right] = \mathcal{L}\left[ \frac{1}{\Gamma(\alpha)} g(t)*f(t) \right] = \frac{1}{\Gamma(\alpha)} \mathcal{L}\left[ g(t) \right]\mathcal{L}\left[ f(t) \right]
\]
And we know that 
\[
    \mathcal{L}\left[ t^{\alpha-1} \right] = \Gamma(\alpha)s^{-\alpha}
\]
Therefore we obtain Laplace Transform of Riemann-Liouville fractional Integral
\begin{equation}
    \mathcal{L}\left[ I^\alpha f(t) \right] = \frac{1}{\Gamma(\alpha)} \Gamma(\alpha)s^{-\alpha}F(s) = s^{-\alpha}F(s)
\end{equation}
\subsubsection{Laplace Transform Of Riemann-Liouville Fractional Derivative}
Now let us try to get Laplace Transform for Riemann-Liouville fractional derivative
\[
    D^{\alpha} f(t) = \dv[m]{}{t} I^{m-\alpha} f(t)
\]
Now take the Laplace Transform of it 
\[
    \mathcal{L}\left[ D^{\alpha} f(t) \right]  = \mathcal{L}\left[ \dv[m]{}{t} I^{m-\alpha} f(t) \right] = \mathcal{L}\left[ \dv[m]{}{t} g(t) \right]
\]
Using the formula for The Laplace Transform of the derivative of an integer order (4.4)
\[
    \mathcal{L}\left[ \dv[m]{}{t} g(t) \right] = s^m \mathcal{L}\left[ g(t) \right] - \sum_{k=0}^{m-1} s^k \left[\dv[m-k-1]{}{t}g(t)\right]_{t=0}
\]
Now Substitute for $g(t) = I^{m-\alpha} f(t)$ we get
\begin{align*}
    \mathcal{L}\left[ D^{\alpha} f(t) \right] &= s^m s^{\alpha-m} F(s) - \sum_{k=0}^{m-1} s^k \left[\dv[m-k-1]{}{t}I^{m-\alpha} f(t)\right]_{t=0}
    \\
    &=s^{\alpha} F(s) - \sum_{k=0}^{m-1} s^k \left[\dv[m-k-1]{}{t}I^{m-k-1-\alpha+k+1} f(t)\right]_{t=0}
    \\
    &=s^{\alpha} F(s) - \sum_{k=0}^{m-1} s^k \left[D^{\alpha-k-1} f(t)\right]_{t=0}
    % &=s^{\alpha} F(s) - s^{m-1} \left[I^{m-\alpha}f(t)\right]_{t=0} - \sum_{k=0}^{m-2} s^k \left[D^{\alpha-k-1} f(t)\right]_{t=0} \tag{4.6}
    \setcounter{equation}{6}
\end{align*}
This is the Laplace transform of the Riemann Liouville fractional derivative
However, it's practical applicability is limited by the absence 
of the physical interpretation of the limit values of fractional derivatives 
at the lower terminal $t = 0$. such an interpretation is not known.
\subsubsection{Laplace Transform Of Caputo Derivative}
The Caputo derivative is defined by
\[
    \leftindex[I]^C{D^\alpha} f(t) = I^{m-\alpha} \dv[m]{}{t} f(t)
\]
Now take the Laplace Transform of both sides
\[
    \mathcal{L}\left[ \leftindex[I]^C{D^\alpha} f(t) \right] = \mathcal{L}\left[ I^{m-\alpha} \dv[m]{}{t} f(t) \right] = s^{\alpha-m} \mathcal{L}\left[ \dv[m]{}{t} f(t) \right]
\]
Using the formula for The Laplace Transform of the derivative of an integer order {\footnotesize(4.3)}
\begin{align*}
    \mathcal{L}\left[ \leftindex[I]^C{D^\alpha} f(t) \right] =s^{\alpha-m}\left[s^m F(s) - \sum_{k=0}^{m-1} s^{m-k-1} f^{(k)}(0) \right]
    = s^{\alpha}F(s) - \sum_{k=0}^{m-1} s^{\alpha-k-1} f^{(k)}(0)
\end{align*}
Since this formula for the Laplace transform of the Caputo derivative
involves the values of the function $f(t)$ and it's derivatives at the lower
terminal $t=0$ for which a certain physical interpretation exists 
(for example, $f(0)$ is the initial position, $f^{(1)}(0)$ is the initial velocity, $f^{(2)}(0)$ is
the initial acceleration), we can expect that it can be useful for solving
applied problems leading to linear fractional differential equations with
constant coefficients with accompanying initial conditions in traditional
form.
\newpage
\subsection{Fourier Transform}
\begin{figure*}[b]
    \begin{minipage}[h]{\textwidth}
        \begin{enrichment}{Jean-Baptiste Joseph Fourier}{Chars/Fourier.jpg}{2.4}{.8}{.17}
            Joseph Fourier (1768–1830) was a brilliant French mathematician and physicist
            although he laid the groundwork for many areas of mathematics, he didn't directly contribute to fractional calculus, 
            a field that emerged later. However, the concept of Fourier transforms  plays a role in modern fractional calculus. 
            Fractional Fourier transforms are mathematical tools used to analyze and solve equations involving non-integer order 
            derivatives.
        \end{enrichment} 
    \end{minipage}
\end{figure*}
The Fourier Transform of a continuous function $f(t)$ absolutely integrable 
on $(-\infty,\infty)$ is defined by
\[
    \mathcal{F}\left[ f(t) \right] = \int_{-\infty}^{\infty} e^{-i\omega t} f(t) \hquad dt = F(\omega)
\]
And the original $f(t)$ can be restored from it's Fourier Transform $F(\omega)$
by the inverse Fourier transform
\[
    \mathcal{F}^{-1}\left[ F(\omega) \right] = \frac{1}{2\pi} \int_{-\infty}^{\infty} e^{i\omega t} F(\omega) \hquad d\omega = f(t)
\]

Fourier Transform have the following properties
\setcounter{property}{0}
\begin{property}
    The Fourier Transform Of The Convolution 

    Let $f(t)$ and $g(t)$ be two functions defined on $(-\infty,\infty)$,
\[
    f(t)*g(t) = \int_{-\infty}^{\infty} f(t-s)g(t) \hquad dt = \int_{-\infty}^{\infty} f(t)g(t-s) \hquad dt    
\]
The Fourier Transform of their convolution is equal to the product of their Fourier transforms
\begin{equation}
    \mathcal{F}\left[ f(t)*g(t) \right] = \mathcal{F}\left[ f(t) \right] \mathcal{F}\left[ g(t) \right] = F(\omega)G(\omega)
\end{equation}
\end{property}
\begin{property}
The Fourier Transform Of Derivatives

Let $f(t)$ and it's derivatives vanish for $t\to \pm\infty$ then 
the Fourier transform of the $ n^{\text{th}} $ derivative of $f(t)$ is
\begin{equation}
    \mathcal{F} \left[ \dv[n]{}{t}f(t) \right] = (-i\omega)^n F(\omega)
\end{equation}
\end{property}
\subsubsection{Fourier Transform Of Riemann-Liouville Integral}
As we did in Laplace Transform here as well 
we will evaluate the Fourier transform of the Riemann-Liouville fractional integral 
but first let's change it's form with lower terminal $-\infty$ and assume that $0<\alpha<1$ we get
\[
    I_{-\infty}^\alpha f(t) = \frac{1}{\Gamma(\alpha)}\int_{-\infty}^{t} (t-s)^{\alpha-1}f(s) \hquad ds
\]
Let us define 
\(
g(t) \begin{cases}
    \displaystyle \frac{t^{\alpha-1}}{\Gamma(\alpha)} \quad &,\quad t>0
    \\
    \displaystyle 0 \quad &,\quad   t \leq 0
\end{cases}
\)

Now to get it's Fourier Transform
\[
    G(\omega) = \mathcal{F}\left[ g(t) \right] = \frac{1}{\Gamma(\alpha)} \int_{0}^{\infty} t^{\alpha-1} e^{-i \omega t} \hquad dt= (-i\omega)^{-\alpha}
\]
Now we can find the Fourier transform of the Riemann-Liouville fractional integral , 
which can be written as a convolution of the functions $g(t)$ and $f(t)$
\[
    \mathcal{F}\left[ I_{-\infty}^\alpha f(t) \right] = \left[ g(t)*f(t) \right] = G(\omega)F(\omega) = (-i\omega)^{-\alpha} F(\omega)
\]
\subsubsection{Fourier Transform Of Riemann-Liouville Derivative}
Now for the Fourier Transform of Riemann-Liouville fractional derivative
\[
    D_{-\infty}^{\alpha} f(t) = \dv[m]{}{t} I_{-\infty}^{m-\alpha} f(t)
\]
Now take the Fourier Transform of it 
\[
    \mathcal{F}\left[ D_{-\infty}^{\alpha} f(t) \right]  = \mathcal{F}\left[ \dv[m]{}{t} I_{-\infty}^{m-\alpha} f(t) \right] = \mathcal{F}\left[ \dv[m]{}{t} g(t) \right]
\]
Using the formula for The Fourier Transform of derivatives (4.8)
\[
    \mathcal{F}\left[ \dv[m]{}{t} g(t) \right] = (-i\omega)^{m} \mathcal{F}\left[ g(t) \right]
\]
Now Substitute for $g(t) = I_{-\infty}^{m-\alpha} f(t)$ we get
\[
\mathcal{F}\left[ D_{-\infty}^{\alpha} f(t) \right] = (-i\omega)^{m} (-i\omega)^{\alpha-m} F(\omega) = (-i\omega)^{\alpha} F(\omega)
\]
\subsubsection{Fourier Transform Of Caputo Derivative}
The Caputo derivative is defined as follows
\[
    \leftindex[I]^C{D_{-\infty}^\alpha} f(t) = I_{-\infty}^{m-\alpha} \dv[m]{}{t} f(t)
\]
Now take the Fourier Transform of it 
\[
    \mathcal{F}\left[ \leftindex[I]^C{D_{-\infty}^\alpha} f(t) \right] = \mathcal{F}\left[ I_{-\infty}^{m-\alpha} \dv[m]{}{t} f(t) \right] = (-i\omega)^{\alpha-m} \mathcal{F}\left[ \dv[m]{}{t} f(t) \right]
\]
Using the formula for The Fourier Transform of derivatives (4.8)
\begin{align*}
    \mathcal{F}\left[ \leftindex[I]^C{D_{-\infty}^\alpha} f(t) \right] &= (-i\omega)^{\alpha-m} \mathcal{F}\left[ \dv[m]{}{t} f(t) \right]
    \\
    &= (-i\omega)^{\alpha-m}(-i\omega)^{m}F(\omega) = (-i\omega)^{\alpha}F(\omega)
\end{align*}
Which is the same as Riemann-Liouville fractional derivative Since both start from $-\infty$
\newpage
\subsection{Mellin Transforms}
The Mellin integral Transform $F(s)$ of a function $f(t)$.
which is defined in the interval $(0,\infty)$ is
\begin{equation}
    F(s) = \mathcal{M}\left[ f(t) \right] = \int_{0}^{\infty} f(t) t^{s-1} \hquad dt
\end{equation}
Where $s$ is complex variable , such as 
\[
\gamma_1 < Re(s) < \gamma_2    
\]
The Mellin transform exists if the function $f(t)$ is piecewise
continuous in every closed interval $[a, b] \subset (0, \infty)$ and
\[
    \int_{0}^{1} |f(t)| t^{\gamma_1-1} \hquad dt < \infty \quad,\quad \int_{1}^{\infty} |f(t)| t^{\gamma_2-1} \hquad dt < \infty
\]
If the function $f(t)$ also satisfies the Dirichlet conditions in every
closed interval $[a, b] \subset (0, \infty)$, then the function $f(t)$ can be restored
using the inverse Mellin transform formula
\[
f(t) = \frac{1}{2\pi i} \int_{\gamma-i\infty}^{\gamma+i\infty} F(s)t^{-s} \hquad ds \quad,\quad (0<t<\infty)
\]

Properties of Mellin transform 
\setcounter{property}{0}
\begin{property}
    Shift Property 
    \\
    It follows from the definition (4.9) that
    \begin{equation}
        \mathcal{M}\left[ t^\alpha f(t) \right] =\mathcal{M}\left[ f(t) \right]_{(s=s+\alpha)}= F(s+\alpha)
    \end{equation}
\end{property}
\begin{property}
The Mellin Transform Of The Mellin Convolution
\[
f(t)*g(t) = \int_{0}^{\infty} f(ts)g(s) \hquad ds
\]
Is given by the formula
\begin{equation}
    \mathcal{M}\left[ \int_{0}^{\infty} f(ts)g(s) \hquad ds \right] = F(s)G(1-s)
\end{equation}
\end{property}
\begin{property}
The Shift Of Convolution
\begin{align*}
    \mathcal{M}\left[ t^\lambda \int_{0}^{\infty} f(ts)g(s) \hquad ds \right] &= \left[ \mathcal{M}\left[\int_{0}^{\infty} f(ts)g(s) \hquad ds \right]\right]_{(s=s+\lambda)}
    \\
    &= F(s)\left[ G(1-s) \right]_{(s=s+\lambda)}
    \\
    &= F(s+\lambda)G(1-s-\lambda)
\end{align*}
\end{property}

\begin{figure*}[b]
    \begin{minipage}[h]{\textwidth}
        \begin{enrichment}{Robert Hjalmar Mellin}{Chars/Mellin.jpg}{2.4}{.8}{.17}
            Hjalmar Mellin (1854 - 1933) was a Finnish mathematician
            His work in the early 1900s laid the foundation for applying fractional derivatives and integrals to real-world problems. 
            Mellin's contributions involved defining the Mellin transform This transformation allowed mathematicians to express 
            fractional-order derivatives and integrals in terms of more familiar operations. 
            Mellin's work opened doors for further development of fractional calculus, 
            making it a valuable tool in various scientific fields.
        \end{enrichment} 
    \end{minipage}
\end{figure*}

\begin{property}
The Mellin Transform Of An Integer Order Derivative
\begin{align*}
    \mathcal{M}\left[ f^{(n)}(t) \right] &= \int_{0}^{\infty} f^{(n)}(t) t^{s-1} dt
    \intertext{Using integration by parts}
    & = \left[ f^{(n-1)}(t) t^{s-1} \right]_0^\infty - (s-1) \int_{0}^{\infty} f^{(n-1)}(t) t^{s-2} dt
    \\
    & = \left[ f^{(n-1)}(t) t^{s-1} \right]_0^\infty - (s-1) \mathcal{M}\left[ f^{(n-1)}(t) \right]_{(s=s-1)}
    \\
    \vdots
    \\
    & = \sum_{k=0}^{n-1}(-1)^k\frac{\Gamma(s)}{\Gamma(s-k)} \left[ f^{(n-k-1)}(t) t^{s-k-1} \right]_0^\infty 
    + (-1)^n \frac{\Gamma(s)}{\Gamma(s-n)} F(s-n)
\end{align*}
But we know that 
\begin{align*}
    \Gamma(x)\Gamma(1-x) &= \frac{\pi}{sin(\pi x)}    
    \intertext{Put $x=s-k$}
    \Gamma(s-k)\Gamma(1-s+k) &= \frac{\pi}{sin(\pi (s-k))}
    \\
    &= \frac{\pi}{sin(\pi s)cos(\pi k)-cos(\pi s)sin(\pi k)}
    \intertext{
        $sin(\pi k) = 0 \quad,\quad cos(\pi k) = (-1)^k$
    }
    &= \frac{\pi}{sin(\pi s)}(-1)^k = \Gamma(s)\Gamma(1-s)(-1)^k
    \intertext{Thus }
    \frac{\Gamma(1-s+k)}{\Gamma(1-s)} &= (-1)^k\frac{\Gamma(s)}{\Gamma(s-k)}
\end{align*}
Therefore 
\begin{equation}
    \mathcal{M}\left[ f^{(n)}(t) \right] =  \sum_{k=0}^{n-1} \frac{\Gamma(1-s+k)}{\Gamma(1-s)} \left[ f^{(n-k-1)}(t) t^{s-k-1} \right]_0^\infty 
    + \frac{\Gamma(1-s+n)}{\Gamma(1-s)} F(s-n)
\end{equation}
If $f(t)$ and $Re(s)$ are such that all substitutions of the limits $t=0$
and $t=\infty$ give zero, then the formula takes it's simplest form
\begin{equation}
    \mathcal{M}\left[ f^{(n)}(t) \right] = \frac{\Gamma(1-s+n)}{\Gamma(1-s)} F(s-n)
\end{equation}
\end{property}
% \newpage
\subsubsection{Mellin Transforms Of Riemann-Liouville Integral}
Let us evaluate the mellin transform of the Riemann Liouville fractional integral.
\begin{align*}
    I^\alpha f(t) &= \frac{1}{\Gamma(\alpha)}\int_{0}^{t} (t-s)^{\alpha-1}f(s) \hquad ds
\intertext{
    Substitute
\(
\begin{cases}
    \displaystyle s = t\theta
    \\
    \displaystyle ds = t \hquad d\theta
    \\
    \displaystyle 0 \to 1
\end{cases}
\)
}
   &= \frac{t^\alpha}{\Gamma(\alpha)}\int_{0}^{1} (1-\theta)^{\alpha-1} f(t\theta) \hquad d\theta
    \intertext{
        Let us define 
\(
g(t) \begin{cases}
    \displaystyle (1-\theta)^{\alpha-1} \quad &,\quad 0 \leq t < 1
    \\
    \displaystyle 0 \quad &,\quad   t \geq 1
\end{cases}
\)
    }
    &= \frac{t^\alpha}{\Gamma(\alpha)}\int_{0}^{\infty} g(\theta) f(t\theta) \hquad d\theta
\end{align*}
Take the Mellin transform for it and using the Shift of Convolution property (3)
\[
    \mathcal{M}\left[ I^\alpha f(t) \right] = \frac{1}{\Gamma(\alpha)} F(s+\alpha)G(1-s-\alpha)
\]
The Mellin transform of the function $g(t)$ gives simply the beta function.
\[
    \mathcal{M}\left[ g(t) \right] = G(s) = \beta(\alpha,s) = \frac{\Gamma(\alpha)\Gamma(s)}{\Gamma(\alpha+s)}
\]
Thus
\[
    \mathcal{M}\left[ I^\alpha f(t) \right] = \frac{1}{\Gamma(\alpha)} F(s+\alpha)\beta(\alpha,1-s-\alpha) = \frac{\Gamma(1-s-\alpha)}{\Gamma(1-s)} F(s+\alpha)
\]
\subsubsection{Mellin Transforms Of Riemann-Liouville Derivative}
Now let us get the Mellin transform of the RL fractional derivative 
\[
    D^{\alpha} f(t) = \dv[m]{}{t} I^{m-\alpha} f(t)
\]
Take the Mellin transform of it 
\begin{align*}
    \mathcal{M}\left[ D^{\alpha} f(t) \right] &= \mathcal{M}\left[ \dv[m]{}{t} I^{m-\alpha} f(t) \right] = \mathcal{M}\left[ \dv[m]{}{t} g(t) \right]
    \intertext{Using The Mellin transform of an integer order derivative property}
    &= \sum_{k=0}^{m-1} \frac{\Gamma(1-s+k)}{\Gamma(1-s)} \left[ \dv[m-k-1]{}{t} g(t) t^{s-k-1} \right]_0^\infty 
    + \frac{\Gamma(1-s+m)}{\Gamma(1-s)} G(s-m)
    \intertext{Now Substitute for $g(t)$}
    &= \sum_{k=0}^{m-1} \frac{\Gamma(1-s+k)}{\Gamma(1-s)} \left[ \dv[m-k-1]{}{t} I^{m-\alpha} f(t) t^{s-k-1} \right]_0^\infty 
    + \frac{\Gamma(1-s+m)}{\Gamma(1-s)}\mathcal{M}\left[ I^{m-\alpha} f(t) \right]_{s=s-m}
    \\
    &= \sum_{k=0}^{m-1} \frac{\Gamma(1-s+k)}{\Gamma(1-s)} \left[ D^{\alpha-k-1} f(t) t^{s-k-1} \right]_0^\infty 
    \\
    &+ \frac{\Gamma(1-s+m)}{\Gamma(1-s)}\left[ \frac{\Gamma(1-s-(m-\alpha))}{\Gamma(1-s)} F(s+(m-\alpha)) \right]_{s=s-m}
    \\
    &= \sum_{k=0}^{m-1} \frac{\Gamma(1-s+k)}{\Gamma(1-s)} \left[ D^{\alpha-k-1} f(t) t^{s-k-1} \right]_0^\infty 
    \\
    &+ \frac{\Gamma(1-s+m)}{\Gamma(1-s)}\frac{\Gamma(1-(s-m)-(m-\alpha))}{\Gamma(1-(s-m))} F((s-m)+(m-\alpha))
    \\
    &= \sum_{k=0}^{m-1} \frac{\Gamma(1-s+k)}{\Gamma(1-s)} \left[ D^{\alpha-k-1} f(t) t^{s-k-1} \right]_0^\infty 
    + \frac{\Gamma(1-s+\alpha)}{\Gamma(1-s)} F(s-\alpha)
    \tag{4.14}
    \setcounter{equation}{14}
\end{align*}
If $0<\alpha<1$ then 
\begin{equation}
    \mathcal{M}\left[ D^{\alpha} f(t) \right] = \left[ I^{1-\alpha} f(t) t^{s-1} \right]_0^\infty + \frac{\Gamma(1-s+\alpha)}{\Gamma(1-s)} F(s-\alpha)
\end{equation}
If $f(t)$ and $Re(s)$ are such that all substitutions of the limits $t=0$
and $t=\infty$ give zero, then the formula takes it's simplest form
\begin{equation}
    \mathcal{M}\left[ D^{\alpha} f(t) \right] = \frac{\Gamma(1-s+\alpha)}{\Gamma(1-s)} F(s-\alpha)
\end{equation}
\subsubsection{Mellin Transforms Of Caputo Derivative}
Let $m-1\leq \alpha < m$ and we know that 
\[
    \leftindex[I]^C{D^\alpha} f(t) = I^{m-\alpha} \dv[m]{}{t} f(t)
\]
Now take the Mellin Transform of it 
\begin{align*}
    \mathcal{M}\left[ \leftindex[I]^C{D^\alpha} f(t) \right] &= \mathcal{M}\left[ I^{m-\alpha} \dv[m]{}{t} f(t) \right] =  \frac{\Gamma(1-s-m+\alpha)}{\Gamma(1-s)} \mathcal{M}\left[ \dv[m]{}{t} f(t) \right]_{s=s+m-\alpha}
    \intertext{
        Using the formula for The Mellin Transform of derivatives (4.13)
    }
    &= \frac{\Gamma(1-s-m+\alpha)}{\Gamma(1-s)} \mathcal{M}\left[ \dv[m]{}{t} f(t) \right]_{s=s+m-\alpha}
    \\
    &= 
    \frac{\Gamma(1-s-m+\alpha)}{\Gamma(1-s)} 
    \left\{ \sum_{k=0}^{m-1} \frac{\Gamma(1-s+k)}{\Gamma(1-s)} \left[ f^{(m-k-1)}(t) t^{s-k-1} \right]_0^\infty 
    \right.
    \\ 
    &\left.+ \frac{\Gamma(1-s+m)}{\Gamma(1-s)} F(s-m)
    \right\}_{s=s+m-\alpha}
    \\
    &= 
    \frac{\Gamma(1-s-m+\alpha)}{\Gamma(1-s)} 
    \left\{ \sum_{k=0}^{m-1} \frac{\Gamma(1-(s+m-\alpha)+k)}{\Gamma(1-(s+m-\alpha))} \left[ f^{(m-k-1)}(t) t^{(s+m-\alpha)-k-1} \right]_0^\infty 
    \right.
    \\ 
    &\left.+ \frac{\Gamma(1-(s+m-\alpha)+m)}{\Gamma(1-(s+m-\alpha))} F((s+m-\alpha)-m)
    \right\}
    \\
    &= \sum_{k=0}^{m-1} \frac{\Gamma(1-(s+m-\alpha)+k)}{\Gamma(1-s)} \left[ f^{(m-k-1)}(t) t^{(s+m-\alpha)-k-1} \right]_0^\infty 
    \\ 
    &+ \frac{\Gamma(1-s+\alpha)}{\Gamma(1-s)} F(s-\alpha)
    \intertext{If we put $k = m-k-1$ the summation value will not change only it's order}
    &= \sum_{k=0}^{m-1} \frac{\Gamma(\alpha-s-k)}{\Gamma(1-s)} \left[ f^{(k)}(t) t^{s+k-\alpha} \right]_0^\infty 
    + \frac{\Gamma(1-s+\alpha)}{\Gamma(1-s)} F(s-\alpha)
    \tag{4.17}
    \setcounter{equation}{17}
\end{align*}
If $0<\alpha<1$ then 
\begin{equation}
    \mathcal{M}\left[ \leftindex[I]^C{D^\alpha} f(t) \right] = \frac{\Gamma(\alpha-s)}{\Gamma(1-s)} \left[ f(t) t^{s-\alpha} \right]_0^\infty + \frac{\Gamma(1-s+\alpha)}{\Gamma(1-s)} F(s-\alpha)    
\end{equation}
If $f(t)$ and $Re(s)$ are such that all substitutions of the limits $t=0$
and $t=\infty$ give zero, then the formula takes it's simplest form
\begin{equation}
    \mathcal{M}\{D^{\alpha} f(t)\} = \frac{\Gamma(1-s+\alpha)}{\Gamma(1-s)} F(s-\alpha)
\end{equation}
