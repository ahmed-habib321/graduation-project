\section{Other Fractional Integral And Derivative Definitions}
In the previous sections we only talked about RL and Caputo Definition of Fractional Integral And Derivative
but there is a lot of other Definitions That we will show some of them now 

\subsection{Riemann Liouville Integral Definitions}
Until now, we considered the fractional integral and derivatives with fixed
lower terminal $a$ and moving upper terminal $t$. 
Moreover we supposed that $a < t$. However, it is also possible to consider fractional integral and derivatives
with moving lower terminal $t$ and fixed upper terminal $b$.
Let us suppose that the function $f(t)$ is defined in the interval $[a, b]$
\begin{enumerate}
        \item The Left Riemann Liouville Fractional Integral
        \begin{equation}
            I_{a+}^{\alpha} f(t) = \frac{1}{\Gamma(\alpha)}\int_{a}^{t} (t-s)^{\alpha-1}f(s) \hquad ds
        \end{equation}
        \item The Right Riemann Liouville Fractional Integral
        \begin{equation}
            I_{b-}^{\alpha} f(t) = \frac{1}{\Gamma(\alpha)}\int_{t}^{b} (s-t)^{\alpha-1}f(s) \hquad ds
        \end{equation}
        \item The Liouville Fractional Integral
        We define the fractional integral according to Liouville by setting $a = -\infty$ , $b = +\infty$
        \begin{equation}
            \leftindex[I]^L {I_{+}^{\alpha}} f(t) = \frac{1}{\Gamma(\alpha)}\int_{-\infty}^{t} (t-s)^{\alpha-1}f(s) \hquad ds
        \end{equation}
        \begin{equation}
            \leftindex[I]^L {I_{-}^{\alpha}} f(t) = \frac{1}{\Gamma(\alpha)}\int_{t}^{+\infty} (s-t)^{\alpha-1}f(s) \hquad ds
        \end{equation}
        \item The Riemann Fractional Integral
        We define the fractional integral according to Riemann by setting $a,b = 0$
        \begin{equation}
            \leftindex[I]^R {I_{+}^{\alpha}} f(t) = \frac{1}{\Gamma(\alpha)}\int_{0}^{t} (t-s)^{\alpha-1}f(s) \hquad ds
        \end{equation}
        \begin{equation}
            \leftindex[I]^R {I_{-}^{\alpha}} = \frac{1}{\Gamma(\alpha)}\int_{t}^{0} (s-t)^{\alpha-1}f(s) \hquad ds
        \end{equation}
\end{enumerate}
This some examples to show the difference Between The Liouville sense and Riemann sense
\begin{align*}
    \leftindex[I]^L {I_{+}^{\alpha}} (e^{kt}) &= k^{-\alpha}e^{kt} && k,t>0
    \\
    \leftindex[I]^L {I_{-}^{\alpha}} (e^{kt}) &= (-k)^{-\alpha}e^{kt} && k<0
    \\
    \leftindex[I]^R {I_{+}^{\alpha}} (e^{kt}) &= (k)^{-\alpha}e^{kt} \left( 1- \frac{\gamma(\alpha,kt)}{\Gamma(\alpha)} \right) && t>0
    \\
    \leftindex[I]^R {I_{-}^{\alpha}} (e^{kt}) &= (-k)^{-\alpha}e^{kt} \left( 1- \frac{\gamma(\alpha,kt)}{\Gamma(\alpha)} \right) && t<0
\end{align*}
Where $\gamma(\alpha,kt)$ is the incomplete gamma function
\begin{enrichment*}{Incomplete Gamma Function}
    The upper and lower incomplete gamma functions are special functions defined by
    
    The upper incomplete gamma function
    \[
    \Gamma(s,x) = \int_{x}^{\infty} t^{s-1} e^{-t} dt
    \]
    The lower incomplete gamma function
    \[
    \gamma(s,x) = \int_{0}^{x} t^{s-1} e^{-t} dt
    \]
\end{enrichment*}
The notions of left and right fractional derivatives can be considered
from the physical and the mathematical viewpoints.
Sometimes the following physical interpretation of the left and right
derivative can be helpful.
Let us suppose that $t$ is time and the function $f(t)$ describes a certain
dynamical process developing in time. If we take $s < t$, where $t$ is the 
present moment, then the state $f(s)$ of the process $f$ belongs to the past
of this process and if we take $s > t$, then $f(s)$ belongs to the future of the process $f$.

\subsection{Riemann Liouville Derivative Definitions}
For the simple case $0 < \alpha < 1$ we obtain
\begin{enumerate}
        \item The Liouville Fractional Derivatives
        \begin{equation}
            \leftindex[I]^L {D_{+}^{\alpha}} f(t) = \dv{}{t} \leftindex[I]^L {I_{+}^{1-\alpha}} f(t) = \frac{1}{\Gamma(1-\alpha)} \dv{}{t} \int_{-\infty}^{t} (t-s)^{-\alpha}f(s) \hquad ds
        \end{equation}
        \begin{equation}
            \leftindex[I]^L {D_{-}^{\alpha}} f(t) = \dv{}{t} \leftindex[I]^L {I_{-}^{1-\alpha}} f(t) = \frac{1}{\Gamma(1-\alpha)} \dv{}{t} \int_{t}^{+\infty} (s-t)^{-\alpha}f(s) \hquad ds
        \end{equation}
        \item The Riemann Fractional Derivatives
        \begin{equation}
            \leftindex[I]^R {D_{+}^{\alpha}} f(t) = \dv{}{t} \leftindex[I]^R {I_{+}^{1-\alpha}} f(t) = \frac{1}{\Gamma(1-\alpha)} \dv{}{t} \int_{0}^{t} (t-s)^{-\alpha}f(s) \hquad ds
        \end{equation}
        \begin{equation}
            \leftindex[I]^R {D_{-}^{\alpha}} f(t) = \dv{}{t} \leftindex[I]^R {I_{-}^{1-\alpha}} f(t) = \frac{1}{\Gamma(1-\alpha)} \dv{}{t} \int_{t}^{0} (s-t)^{-\alpha}f(s) \hquad ds
        \end{equation}
        \item The Liouville-Caputo Fractional Derivatives
        \begin{equation}
            \leftindex[I]^{LC} {D_{+}^{\alpha}} f(t) = \leftindex[I]^L {I_{+}^{1-\alpha}} \dv{}{t} f(t) = \frac{1}{\Gamma(1-\alpha)}  \int_{-\infty}^{t} (t-s)^{-\alpha} \dv{f(s)}{s}  \hquad ds
        \end{equation}
        \begin{equation}
            \leftindex[I]^{LC} {D_{-}^{\alpha}} f(t) = \leftindex[I]^L {I_{-}^{1-\alpha}} \dv{}{t} f(t) = \frac{1}{\Gamma(1-\alpha)} \int_{t}^{+\infty} (s-t)^{-\alpha} \dv{f(s)}{s} \hquad ds
        \end{equation}
        \item The Riemann-Caputo Fractional Derivative (The Caputo Fractional Derivative)
        \begin{equation}
            \leftindex[I]^{C} {D_{+}^{\alpha}} f(t) = \leftindex[I]^R {I_{+}^{1-\alpha}} \dv{}{t} f(t) = \frac{1}{\Gamma(1-\alpha)}  \int_{0}^{t} (t-s)^{-\alpha} \dv{f(s)}{s}  \hquad ds
        \end{equation}
        \begin{equation}
            \leftindex[I]^{C} {D_{-}^{\alpha}} f(t) = \leftindex[I]^R {I_{-}^{1-\alpha}} \dv{}{t} f(t) = \frac{1}{\Gamma(1-\alpha)} \int_{t}^{0} (s-t)^{-\alpha} \dv{f(s)}{s} \hquad ds
        \end{equation}
\end{enumerate}
\subsection{Differ-Integral Operator}
If we combine fractional differentiation and fractional 
integration we get a unified derivative-integral operator 
that works basically as a piecewise combination of the two where 
plugging in a positive order uses the fractional derivative 
formula, and plugging in a negative order uses the fractional 
integral formula. This combined operator is known as a 
"Differ-Integral" and it's defined by
\[
    \prescript{RL}{a}{D_{t}^{\alpha}} f(t) = 
    \begin{cases}
        \displaystyle D^{\alpha} f(t) = \frac{1}{\Gamma(m-\alpha)} \dv[m]{}{t}\int_{a}^{t} (t-s)^{m - \alpha -1}f(s) \hquad ds \qquad &\text{if } \alpha > 0
        \\
        \displaystyle f(t)  &\text{if } \alpha = 0
        \\
        \displaystyle I^{\alpha} f(t) = \frac{1}{\Gamma(\alpha)}\int_{a}^{t} (t-s)^{\alpha -1}f(s) \hquad ds &\text{if } \alpha < 0
    \end{cases}
\]
\subsection{Fractional Derivative According To Fourier}
The Fractional Derivative represented using Fourier series is 
\[
    f(x) = a_0 + \sum_{k=1}^{\infty} a_k cos(kx) + b_k sin(kx)
\]
\[
    \leftindex[I]^{\mathcal{F}} {D^{\alpha}} f(x) = \sum_{k=1}^{\infty} a_k k^\alpha cos(kx + \frac{\pi}{2} \alpha) + b_k k^\alpha sin(kx + \frac{\pi}{2} \alpha)
\]
\newpage
\subsection{The Grünwald-Letnikov Fractional Derivative}
Another definition of a fractional derivative in terms of a limit of finite differences.
Starting with the definition of the first derivative as
\[
\dv{}{t} f(t) = \lim_{h \to 0} \frac{\Delta f(x)}{h} = \lim_{h \to 0} \frac{f(x)-f(x-h)}{h}
\]
And for higher derivatives
\begin{align*}
    \dv[2]{}{t} f(t) &= \lim_{h \to 0} \frac{\Delta^2 f(x)}{h^2} = \lim_{h \to 0} \frac{\Delta (f(x)-f(x-h))}{h^2}
    \\
    &=\lim_{h \to 0} \frac{f(x)-2f(x-h)+f(x-2h)}{h^2}
    \\
    \dv[3]{}{t} f(t) &= \lim_{h \to 0} \frac{\Delta^3 f(x)}{h^3} = \lim_{h \to 0} \frac{\Delta^2 (f(x)-f(x-h))}{h^3}
    \\
    &=\lim_{h \to 0} \frac{\Delta(f(x)-2f(x-h)+f(x-2h))}{h^3}
    \\
    &=\lim_{h \to 0} \frac{f(x)-3f(x-h)+3f(x-2h)-f(x-3h)}{h^3}
\end{align*}
And so on we can deduce the formula for the $n^{\text{th}}$ derivative
\[
    \dv[n]{}{t} f(t) = \lim_{h \to 0} \frac{\Delta^n f(x)}{h^n} 
    = \lim_{h \to 0} \left[  \frac{1}{h^{n}} \sum_{k=0}^{n} (-1)^{k} {n \choose k} f(t-kh) \right]
\]
Where $\displaystyle {n \choose k} = \frac{n!}{k!(n-k)!}$ if we change the factorial to the gamma function 
we reach The Grünwald-Letnikov Fractional Derivative
\begin{align*}
    \leftindex[I]^{GL} {D^{\alpha}} f(t) 
    &= \lim_{h \to 0} \frac{\Delta^\alpha f(x)}{h^\alpha} 
    \\
    &= \lim_{h \to 0} \left[  \frac{1}{h^{\alpha}} \sum_{k=0}^{\infty} (-1)^{k} {\alpha \choose k} f(t-kh) \right]
    \\
    &= \lim_{h \to 0} \left[  \frac{1}{h^{\alpha}} \sum_{k=0}^{\infty} (-1)^{k} \frac{\Gamma(\alpha)}{k!\Gamma(\alpha-k)} f(t-kh) \right]
\end{align*}
\subsection{The Canavati Fractional Derivative}
There is another definition of fractional derivatives that is useful in deriving
inequalities. This is the Canavati fractional derivative. It is "between" the
Riemann-Liouville derivative and the Caputo derivative. 

Let $m-1 < \alpha < m$. Then,
the Canavati derivative of order $\alpha$ is defined as
\[
    \prescript{Can}{a}{D_{t}^{\alpha}} f(t) = \frac{1}{\Gamma(m-\alpha)}\dv{}{t}\int_{a}^{t} (t-s)^{n-\alpha-1} \dv[n-1]{}{s}f(s)ds
\]
For $f(t) \in C^\alpha[a,b]$
\[
    C^\alpha[a,b] := \left\{ f \in C^{m-1}[a,b] \quad | \quad \leftindex[I]_a {I_{t}^{n-1}} f(t) \in C^{1}[a,b] \right\}
\]
%%%%%%%%%%%%%%%%%%%%%%%%%%%%%%%%%%%%%%%%%%%%%%%%%%%%%%%%%%%%
%%%%%%%%%%%%%%%%%%%%%%%%%%%%%%%%%%%%%%%%%%%%%%%%%%%%%%%%%%%%
%%%%%%%%%%%%%%%%%%%%%%%%%%%%%%%%%%%%%%%%%%%%%%%%%%%%%%%%%%%%
%%%%%%%%%%%%%%%%%%%%%%%%%%%%%%%%%%%%%%%%%%%%%%%%%%%%%%%%%%%%
%%%%%%%%%%%%%%%%%%%%%%%%%%%%%%%%%%%%%%%%%%%%%%%%%%%%%%%%%%%%
%%%%%%%%%%%%%%%%%%%%%%%%%%%%%%%%%%%%%%%%%%%%%%%%%%%%%%%%%%%%
%%%%%%%%%%%%%%%%%%%%%%%%%%%%%%%%%%%%%%%%%%%%%%%%%%%%%%%%%%%%
%%%%%%%%%%%%%%%%%%%%%%%%%%%%%%%%%%%%%%%%%%%%%%%%%%%%%%%%%%%%
% محتاج اعرف الهولدر سبيس عشان هستعمله في التعريف الجي
%%%%%%%%%%%%%%%%%%%%%%%%%%%%%%%%%%%%%%%%%%%%%%%%%%%%%%%%%%%%
%%%%%%%%%%%%%%%%%%%%%%%%%%%%%%%%%%%%%%%%%%%%%%%%%%%%%%%%%%%%
%%%%%%%%%%%%%%%%%%%%%%%%%%%%%%%%%%%%%%%%%%%%%%%%%%%%%%%%%%%%
%%%%%%%%%%%%%%%%%%%%%%%%%%%%%%%%%%%%%%%%%%%%%%%%%%%%%%%%%%%%
%%%%%%%%%%%%%%%%%%%%%%%%%%%%%%%%%%%%%%%%%%%%%%%%%%%%%%%%%%%%
%%%%%%%%%%%%%%%%%%%%%%%%%%%%%%%%%%%%%%%%%%%%%%%%%%%%%%%%%%%%
%%%%%%%%%%%%%%%%%%%%%%%%%%%%%%%%%%%%%%%%%%%%%%%%%%%%%%%%%%%%
%%%%%%%%%%%%%%%%%%%%%%%%%%%%%%%%%%%%%%%%%%%%%%%%%%%%%%%%%%%%
%%%%%%%%%%%%%%%%%%%%%%%%%%%%%%%%%%%%%%%%%%%%%%%%%%%%%%%%%%%%
\subsection{The Marchaud Fractional Derivative}
The Left Marchaud fractional derivative of the order $0 < \alpha < 1$ for $f(t) \in \mathcal{H}^\lambda[a,b] $ , 
$\lambda>\alpha$ defined by
\[
    \prescript{M}{a}{D_{t}^{\alpha}} f(t) = \frac{f(t)}{\Gamma(1-\alpha)(t-a)^\alpha} 
    + \frac{\alpha}{\Gamma(1-\alpha)} \int_{a}^{t} \frac{f(t) - f(s)}{(t-s)^{1+\alpha}} \hquad ds
\]
The Right Marchaud fractional derivative is defined as
\[
    \prescript{M}{t}{D_{b}^{\alpha}} f(t) = \frac{f(t)}{\Gamma(1-\alpha)(b-t)^\alpha} 
    + \frac{\alpha}{\Gamma(1-\alpha)} \int_{t}^{b} \frac{f(t) - f(s)}{(s-t)^{1+\alpha}} \hquad ds
\]
\newpage
\subsection{The Riesz Fractional Integral And Derivative}
The most prominent approach is the Riesz fractional integral and fractional derivative

It is a linear combination of both left and right fractional Liouville integrals (5.3) and (5.4)
\begin{align*}
    \leftindex[I]^{RZ} {I^{\alpha}} f(t) &= \frac{\leftindex[I]^L {I_{+}^{\alpha}} + \leftindex[I]^L {I_{-}^{\alpha}}}{2 \cos(\frac{\pi}{2}\alpha)} f(t)
    \\
    & = \frac{1}{2 \Gamma(\alpha) \cos(\frac{\pi}{2}\alpha)}\int_{-\infty}^{+\infty} |t-s|^{\alpha-1}f(s) \hquad ds
\end{align*}
This is fractional Riesz integral

In order to derive the explicit form of the Riesz fractional derivative
we first present the left and right Liouville derivative (5.7) and (5.8) in an alternative form
\begin{align*}
    \leftindex[I]^L {D_{+}^{\alpha}} f(t) &= \frac{\alpha}{ \Gamma(1-\alpha)} \int_{0}^{\infty} \frac{f(t)-f(t-s)}{s^{\alpha+1}} ds
    \\
    \leftindex[I]^L {D_{-}^{\alpha}} f(t) &= \frac{\alpha}{ \Gamma(1-\alpha)} \int_{0}^{\infty} \frac{f(t)-f(t+s)}{s^{\alpha+1}} ds
\end{align*}
Which follows from (5.7)
\begin{align*}
    \leftindex[I]^L {D_{+}^{\alpha}} f(t) &= \dv{}{t} \leftindex[I]^L {I_{+}^{1-\alpha}} f(t) 
    \\
    &= \frac{1}{\Gamma(1-\alpha)} \dv{}{t} \int_{-\infty}^{t} (t-s)^{-\alpha}f(s) \hquad ds
    \intertext{
            Substitute
    \(
    \begin{cases}
        \displaystyle (t-s) = \xi
        \\
        \displaystyle d\xi = -ds
        \\
        \displaystyle \infty \to 0
    \end{cases}
    \)
        }
        &= \frac{1}{\Gamma(1-\alpha)} \dv{}{t} \int_{\infty}^{0} \xi^{-\alpha}f(t-\xi) \hquad (-d\xi)
        \\
        &= \frac{1}{\Gamma(1-\alpha)} \int_{0}^{\infty} \xi^{-\alpha}\pdv{}{t}f(t-\xi) \hquad d\xi
        \\
        &= \frac{1}{\Gamma(1-\alpha)} \int_{0}^{\infty} \xi^{-\alpha}\left(-\pdv{}{\xi}f(t-\xi)\right) \hquad d\xi
        \\
        &= \frac{\alpha}{\Gamma(1-\alpha)} \left(\int_{0}^{\infty} \frac{f(t)}{\xi^{\alpha+1}} \hquad d\xi - \int_{0}^{\infty} \frac{f(t-\xi)}{\xi^{\alpha+1}} \hquad d\xi \right)
        \\
        &= \frac{\alpha}{\Gamma(1-\alpha)} \int_{0}^{\infty} \frac{f(t) - f(t-\xi)}{\xi^{\alpha+1}} \hquad d\xi 
\end{align*}
And similarly for (5.8)

Now applying the formula
\[
    \Gamma(\alpha)\Gamma(1-\alpha) = \frac{\pi}{\sin(\pi \alpha)}
\]
We get 
\[
    \frac{\alpha}{\Gamma(1-\alpha)} = \Gamma(1+\alpha) \frac{\sin(\pi \alpha)}{\pi}
\]
With the definition of the fractional derivative according to Riesz
\[
\leftindex[I]^{RZ} {D^{\alpha}} f(t) = - \frac{\leftindex[I]^L {D_{-}^{\alpha}} + \leftindex[I]^L {D_{+}^{\alpha}}}{2 \cos(\frac{\pi}{2}\alpha)} f(t)
\]
We explicitly obtain
\begin{align*}
    \leftindex[I]^{RZ} {D^{\alpha}} f(t) &= \Gamma(1+\alpha) \frac{\sin(\pi \alpha)}{2\pi \cos(\frac{\pi}{2}\alpha)} \int_{0}^{\infty} \frac{f(t+s) -2f(t) + f(t-s)}{s^{\alpha+1}} \hquad ds
    \\
    &= \Gamma(1+\alpha) \frac{\sin(\frac{\pi}{2}\alpha)}{\pi} \int_{0}^{\infty} \frac{f(t+s) -2f(t) + f(t-s)}{s^{\alpha+1}} \hquad ds
\end{align*}


\subsection{The Feller Fractional Integral And Derivative}
A possible generalization for the Riesz fractional derivative was proposed
by Feller (1952). He suggested a general superposition of both
fractional Liouville integrals (5.3) and (5.4)
\[
    \leftindex[I]^{F} {I_{\theta}^{\alpha}} = c_-(\theta,\alpha) \leftindex[I]^L {I_{+}^{\alpha}} + c_+(\theta,\alpha) \leftindex[I]^L {I_{-}^{\alpha}}
\]
Introducing a free parameter 0 < $\theta$ < 1 which is a measure for the influence of both components
\begin{align*}
    c_-(\theta,\alpha) = \frac{\sin(\frac{\pi}{2}(\alpha-\theta))}{\sin(\pi \theta)}
    \\
    c_+(\theta,\alpha) = \frac{\sin(\frac{\pi}{2}(\alpha+\theta))}{\sin(\pi \theta)}
\end{align*}
The fractional Feller derivative is then given as
\[
    \prescript{F}{}{D_{\theta}^{\alpha}} = - \left[c_+(\theta,\alpha) \prescript{}{L}{D_{+}^{\alpha}} + c_+(\theta,\alpha) \prescript{}{L}{D_{-}^{\alpha}}\right]
\]
For the special case $\theta$ = 0 we obtain
\[
    c_+(0,\alpha) = c_-(0,\alpha) = \frac{1}{2 \cos(\alpha \frac{\pi}{2})}
\]
Which exactly corresponds to the definition of the Riesz derivative.
\subsection{Hadamard Fractional Integral And Derivative}
The Hadamard fractional Integral of order $\alpha$ of a function $f(t) \in L_p[a,b]$ and $a \neq 0$ is defined by 
\[
    \leftindex[I]^H {I_a^{\alpha}} f(t) = \frac{1}{\Gamma(\alpha)} \int_{a}^{t} \left(\ln \frac{t}{s}\right)^{\alpha-1} \frac{f(s)}{s} \hquad ds
\]
Let $m-1 < \alpha < m$ the Hadamard fractional derivative of order $\alpha$
of a function $f(t) \in AC^m[a,b]$ is defined by 
\[
    \leftindex[I]^H {D_a^{\alpha}} f(t) = \frac{1}{\Gamma(m-\alpha)} \dv[m]{}{t} \int_{a}^{t} \left(\ln \frac{t}{s}\right)^{n-\alpha-1} \frac{f(s)}{s} \hquad ds
\]
And the Hadamard-Caputo fractional derivative of order $\alpha$ is defined by 
\[
    \leftindex[I]^{HC} {D_a^{\alpha}} f(t) = \frac{1}{\Gamma(m-\alpha)} \int_{a}^{t} \left(\ln \frac{t}{s}\right)^{n-\alpha-1} \frac{f^{(m)}(s)}{s} \hquad ds
\]
\newpage
\subsection{Generalized Fractional Integration And Differentiation}

When we used the repeated integration formula we used 
Riemann definition of the integration but if we use Riemann-Stieltjes Integral
\[
    \mathcal{J} f(t) = \int_{0}^{t} f(s) \hquad dg(s)
\]
We get 
\begin{align*}
    \mathcal{J} f(t) &= \int_{0}^{t} f(s) \hquad dg(s)
    \\
    \mathcal{J}^2 f(t) &= \int_{0}^{t}\int_{0}^{s} f(\theta) \hquad dg(\theta) \hquad dg(s)
    \intertext{We can interchanging the order of integration using Fubini’s theorem}
    &= \int_{0}^{t}\int_{\theta}^{t} \hquad dg(s) \hquad f(\theta) \hquad dg(\theta)
    \\
    &= \int_{0}^{t}(g(t)-g(\theta)) f(\theta) \hquad dg(\theta) = \int_{0}^{t}(g(t)-g(s)) f(s) \hquad dg(s)
    \\
    \mathcal{J}^3 f(t) &= \int_{0}^{t} \int_{0}^{s} (g(s)-g(\theta)) f(\theta) \hquad dg(\theta) \hquad dg(s)
    \\
    &= \int_{0}^{t} \int_{\theta}^{s} (g(s)-g(\theta)) \hquad dg(s) \hquad f(\theta) \hquad dg(\theta)
    \\
    &= \int_{0}^{t} \frac{(g(t)-g(\theta))^2}{2} f(\theta) \hquad dg(\theta)
    \intertext{And so on we get that}
    \mathcal{J}^n f(t) &=\int_{0}^{t} \frac{(g(t)-g(s))^{n-1}}{(n-1)!} f(s) \hquad dg(s)
\end{align*}
Now change the factorial to the Gamma function and the order $n$ to an arbitrary order $\alpha$ we get the Generalized Fractional integral
\[
    \mathcal{J}^\alpha f(t) =\frac{1}{\Gamma(\alpha)} \int_{0}^{t} (g(t)-g(s))^{\alpha-1} f(s) \hquad dg(s)
\]
Or
\[
    \mathcal{J}^\alpha f(t) =\frac{1}{\Gamma(\alpha)} \int_{0}^{t} (g(t)-g(s))^{\alpha-1}g'(s) f(s) \hquad ds
\] 
In case of RL Fractional integral $g(t)$ is chosen to be $t$ and in hadamard is chosen to be $ln(t)$
% g is increasing and 
%   - f continuous 
%   - f monotonic + g continuous
and using this definition we can deduce the Generalized Fractional Derivative of Riemann sense
\[
    \mathfrak{D}^\alpha f(t) =\dv[m]{}{t}\mathcal{J}^{m-\alpha} f(t) 
    = \frac{1}{\Gamma(m-\alpha)} \dv[m]{}{t}\int_{0}^{t} (g(t)-g(s))^{m-\alpha-1} f(s) \hquad dg(s)
\] 
And Caputo sense
\[
    \leftindex[I]^C \mathfrak{D}^\alpha f(t) =\mathcal{J}^{m-\alpha} \dv[m]{}{t}f(t) 
    = \frac{1}{\Gamma(m-\alpha)} \int_{0}^{t} (g(t)-g(s))^{m-\alpha-1} \dv[m]{f(s)}{s} \hquad dg(s)
\] 
\newpage
\subsection{Kolwankar And Gangal (1994)}
Local fractional calculus is a new branch of mathematics (is also called Fractal calculus) 
was first introduced by Kolwankar and Gangal. It deals with derivatives and integrals of the functions defined on fractal sets. 
And it is explain the behavior of continuous but nowhere differentiable function.

They proposed the following definition for the local fractional derivative of a function defined on fractal sets
\[
    \prescript{KG}{x_0}{D_x^{\alpha}} f(x) = \lim_{x \to x_0}  \dv[\alpha]{(f(x)-f(y))}{(x-y)} \quad,\quad 0<\alpha<1
\]
And the local fractional integrals of a function defined on fractal sets is defined as   
\[
    \prescript{KG}{a}{I_b^{\alpha}} f(x) = \lim_{N \to \infty} \sum_{i=0}^{N-1} f(x_i^*) \frac{d^{-\alpha} 1_{dx_i(x)}}{d(x_{i+1}-x_i)}
\]
Where $1_{dx_i(x)}$ is the unit function defined upon $[x_{i},x_{i+1}]$ and $x_i^*$ is a point in the interval $[x_{i},x_{i+1}]$
and the intervals $[x_{i},x_{i+1}]$ , $i=0,1,2,\dots,N-1$ are partitions of the interval $[a,b]$

\subsection{The Gohar Fractional Integral And Derivative}
And last but not least The Gohar definition a new local fractional derivative
introduced by Abdelrahman Gohar ,Mayada Younes and Salah B.Doma in 2023
which generalizes the classical limit definition of the derivative.

Given a function $f : [0,\infty) \to \mathbb{R}$, the GFD of $f$ of order $\alpha$, denoted by $G_\alpha$ is defined by
\[
    G_\alpha f(t) = \lim_{h \to 0} \frac{1}{h}\left[  f\left( t \left\{ 1+ \ln\left( 1+ \frac{h\Gamma(\eta)}{\Gamma(\eta-\alpha+1)} t^{-\alpha} \right)\right\}\right) -f(t) \right]
\]
For $t>0$ , $\alpha \in (0,1)$ , $\eta \in \mathbb{R}^+$

And for $t \geq 0$ if $f$ is a function defined on $(0, t]$, then the GFI of $f$, of order $\alpha$,
is defined by
\[
    \mathfrak{T}^\alpha f(t) = \frac{\Gamma(\eta-\alpha+1)}{\Gamma(\eta)} \int_{0}^{t} \frac{f(s)}{s^{1-\alpha}} ds
\]

And there is a lot of others different formulations for fractional derivatives like
\begin{itemize}
    \item Chen Derivative
    \item Cossar Derivative
    \item Davidson-Essex Derivative
    \item Hilfer Derivative
    \item Jumarie Derivative
    \item Osler Derivative
    \item Weyl Derivative
    \item Yang Derivative
\end{itemize}
And there's no definitive version because each one of them was trying to preserve some of the ordinary derivative 
properties but there is no such Definition for the fractional derivative that preserve all the ordinary derivative 
properties